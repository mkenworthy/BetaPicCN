%                                                                 aa.dem
% AA vers. 8.2, LaTeX class for Astronomy & Astrophysics
% demonstration file
%                                                       (c) EDP Sciences
%-----------------------------------------------------------------------
%
%\documentclass[referee]{aa} % for a referee version
%\documentclass[onecolumn]{aa} % for a paper on 1 column  
%\documentclass[longauth]{aa} % for the long lists of affiliations 
%\documentclass[rnote]{aa} % for the research notes
%\documentclass[letter]{aa} % for the letters 
%\documentclass[bibyear]{aa} % if the references are not structured 
% according to the author-year natbib style
\documentclass{aa}  
\usepackage{rotating}
\usepackage{natbib}
\usepackage{amsmath}
\bibpunct{(}{)}{;}{a}{}{,} % to follow the A&A style

\newcommand{\kms}{km s$^{-1}$}
\newcommand{\bpb}{Beta Pictoris b}
\newcommand{\bp}{Beta Pictoris}

\usepackage{color}
\usepackage{hyperref}
\hypersetup{colorlinks=true,allcolors=[rgb]{0,0,0.8}}

\usepackage[nolist,nohyperlinks]{acronym}
\newacro{harps}[HARPS]{High Accuracy Radial Velocity Planet Searcher}
\newacro{uves}[UVES]{Ultra-violet Visible Echelle Spectrograph}
\newacro{feb}[FEBs]{Falling Evaporating Bodies}

%
\usepackage{graphicx}
%%%%%%%%%%%%%%%%%%%%%%%%%%%%%%%%%%%%%%%%
\usepackage{txfonts}
%%%%%%%%%%%%%%%%%%%%%%%%%%%%%%%%%%%%%%%%
%\usepackage[options]{hyperref}
% To add links in your PDF file, use the package "hyperref"
% with options according to your LaTeX or PDFLaTeX drivers.
%
\usepackage{showyourwork}

% text highlighting
\usepackage{soul}
\sethlcolor{yellow}

\usepackage[version=4]{mhchem}
% the three lines suppress the hyperref 'link empty' warnings
% explanation at: https://tex.stackexchange.com/questions/345764/journal-class-shows-package-hyperref-warning-suppressing-link-with-empty-targe
\makeatletter
\renewcommand*\aa@pageof{, page \thepage{} of \pageref*{LastPage}}
\makeatother

\begin{document} 


   \title{Upper Limits on CN from Falling Evaporating Bodies\\seen towards \bp{}}

   \author{M.A. Kenworthy
          \inst{1}
          \and
          E. de Mooij\inst{2}
          \and
          C. Opitom\inst{3}
          \and
          A.\ Brandeker\inst{4}
          \and 
          F. Kiefer\inst{5}
          \and
          A. Fitzsimmons \inst{1}
          }

   \institute{Leiden Observatory, Leiden University, P.O. Box 9513, 2300 RA Leiden, The Netherlands\\
              \email{kenworthy@strw.leidenuniv.nl}
         \and
             Astrophysics Research Centre, School of Mathematics and Physics, Queen’s University Belfast, BT7 1NN Belfast, UK
        \and
             Institute for Astronomy, University of Edinburgh, Royal Observatory, Edinburgh EH9 3HJ, UK
        \and
        Institutionen f\"{o}r astronomi, Stockholms universitet, AlbaNova universitetscentrum, 106 91, Stockholm, Sweden
        \and
        LESIA, Observatoire de Paris, Université PSL, CNRS, 5 Place Jules Janssen, 92190 Meudon, France}

   \date{Received XXXX; Accepted XXXX}

% \abstract{}{}{}{}{} 
% 5 {} token are mandatory
 
  \abstract
  % context heading (optional)
  % {} leave it empty if necessary  
   {Beta Pic has \ac{feb} which are seen as variable transient absorption features towards \bp{}.}
  % aims heading (mandatory)
   {To detect cyanogen (\ce{CN}) in the \ac{feb} spectrum, confirming that they are similar to comets in the Solar System.}
  % methods heading (mandatory)
   {We combine the \ac{harps} spectra from the past 20 years to make a high signal to noise ratio spectrum and look for the \ce{CN} band head. We also combine \ac{uves} data to confirm our non-detections/detections.}
  % results heading (mandatory)
   {No \ce{CN} is seen to a level of XXX ppt, indicating that there is less \ce{CN} in these \ac{feb} than Solar system comets.}
  % conclusions heading (optional), leave it empty if necessary 
   {\ac{feb} are not Solar System comets in composition.}

   \keywords{comets --- transits, spectroscopy}

   \maketitle
%
%________________________________________________________________

\section{Introduction}

The formation and evolution of planets and their attendant moons is thought to occur in the first few ten million years of the stellar system \hl{REF}.
%
Comets deliver volatiles across menay different distances within the Solar system, from the outermost reaches of the Oort cloud in to the terrestrial planets.
%
It is therefore important to understand the distribution, nature and chemical composition of exocomets in other planet forming systems.

One of the closest young systems is Beta Pictoris, which has been studied intensively both with direct imaging and with spectroscopy.

Beta Pictoris is a young, early type star whose stellar chromosphreic lines show rotational broadening of 200 km/s \hl{REF}.
%
Additional, much narrower absorption features are seen in the Calcium H and K lines.
%
These features appear and disappear on several hour long timescales, and show a different range of morphologies \hl{[REF]}.
%
Intensive spectroscopic monitoring over several hours show changes of radial velocity of these components consistent with elliptical orbits.
%
The absorption is hypothesised to be due to the evaporation of infalling bodies that pass within a few stellar radii of the star, resulting in sublimation of metals that produce a large coma containing the Calcium atoms which are then seen as an additional absorption component \hl{[REF REF]}.

From the multiple detections of exocomets in spectroscopy towards Beta Pictoris, the broadband detection of exocomet tails transiting a star were seen initially detected towards the star KIC~3542116 \citep{Rappaport18} with the Kepler satellite [REF] and subsequently detected in TESS data towards Beta Pictoris \citep{Zieba19}.
%
Observations of Beta Pictoris in subsequent TESS sector revealed over two dozen transits whost depth follows a power law distribution similar to those of the Solar system \citep{LecavelierdesEtangs22,Pavlenko22}.



%Short Beta pic description, previous detection of species in optical (CaII, NaI, FeI).


To the contrary of what is observed in the case of \ac{feb}, most molecular and atomic signatures detected in the coma of Solar system comets are emission lines/bands instead of absorptions.

Spectroscopic observations of Solar system comets when they are very close to the Sun are rare due to observational difficulties.
%
%The only comet spectroscopically observed closer than 0.2~au from the Sun is C/1965~S1 Ikeya-Seki.
%
%For this comet, it was noticed that the main emission bands/lines detected in its optical spectrum dramatically changed while the comet approached the Sun.
%
%At distances greater than 0.4~au, \ce{CH}, \ce{CN}, \ce{C2}, and \ce{C3} emission bands dominated the spectrum, but at 0.15 au from the Sun, the only molecular emission detected was \ce{CN}, while numerous atomic lines were visible \citep[\ce{[\ion{O}{i}], \ion{Na}{i}, \ion{K}{i}, \ion{Ca}{ii}, \ion{Cr}{i}, \ion{Mn}{i}, \ion{Fe}{i}, \ion{Ni}{i}, \ion{Cu}{i}, \ion{V}{i}; ][]{Dufay65,Thackeray1966,Preston1967,Slaughter1969}.
%%\ion{Ca}{ii} and \ion{Fe}{i} have only been spectroscopically observed via emission lines in the sun-grazing comet C/1965~S1 Ikeya-Seki.
%
For this comet, it was noticed that the main emission bands/lines detected in its optical spectrum dramatically changed while the comets approached the Sun.
%
At distances greater than 0.4~au, \ce{CH}, \ce{CN}, \ce{C2}, and \ce{C3} emission bands dominated the spectrum.
%
However, at 0.15~au from the Sun, the only molecular emission detected was \ce{CN}, while numerous atomic lines were visible \citep[\ion{[O]}{i}, \ion{Na}{i}, \ion{K}{i}, \ion{Ca}{ii}, \ion{Cr}{i}, \ion{Mn}{i}, \ion{Fe}{i}, \ion{Ni}{i}, \ion{Cu}{i}, \ion{V}{i}; ][]{Dufay65,Thackeray1966,Preston1967,Slaughter1969}.
%
Additionally, a tail of \ion{Fe}{i} atoms was proposed to explain a linear feature imaged in the tail of sungrazing comet C/2016~P1 McNaught 2006P1 \citep{Fulle2007}. 

On the other hand, \ion{Na}{i} D-line emission has been observed in many comets within $r \sim 1$ au of the Sun, where either the emission was much brighter than the foreground sky emission, or the comet was observed at high spectral resolution enabling separation from the telluric lines \hl{REFs}.
%
The high efficiency of the \ion{Na}{i} D transition makes it easily detectable, even for relatively low column density.
%
A notable discovery was a spectacular tail of \ion{Na}{i} associated with comet C/1995 O1 Hale-Bopp at $r\sim 1$ au, formed by radiation pressure on the neutral atoms before ionisation \citep{Cremonese1997}. 

These characteristics imply that \ion{Ca}{ii} and \ion{Fe}{i} are predominantly locked up within cometary dust gains, and only observed when a comet is close enough to the Sun to allow their sublimation.
%
This occurs at $r\leq 0.03-0.06$ au for a dust grain with Bond albedo $\sim0.01$ and a sublimation temperature of $T\simeq 1600$K, depending on the thermal model assumed.
%
On the other hand, the origin of the \ion{Na}{i} tail observed in numerous comets at larger distances from the Sun is still not very well understood \citep{Cremonese02}.
%
Release from dust grains as well as molecular processes have been suggested as possible sources of \ion{Na}{i} in the coma and tail of comets annd explain its presence at large solar distances \citep{Cremonese1997}.
%Na appears to be weakly bound to the dust grains and released at larger solar distances.

For the majority of Solar system comets, their optical spectra are dominated by the reflected sunlight from dust grains in their comae, plus molecular and atomic emission lines \citep[e.g. ][]{hyland2019}.
%
Most of the neutral and ionic molecular emission results from resonance fluorescence with solar photons. Some species, such as \ce{H2O+} and \ce{CO+} result from ionization of parent molecules (i.e. molecules released directly by the sublimation of nuclear ices), all other molecules observed are second generation species, often created via photodissociation of parent molecules or released by organic-rich grains in the coma.
%
Another process also observed in optical spectrum of comets is prompt atomic emission  ([\ion{O}{i}], [\ion{C}{i}], and [\ion{N}{i}]).
%
Those forbidden transitions are the result of atoms produced in an excited state by the photodissociation of parent molecules in the coma. 

Of all cometary gas emission, the highest signal to noise normally occurs with the \ce{CN} (0-0) band at $\lambda\simeq 3880$\AA.
%
It is usually the first emission detected in the optical spectrum of a comet as it approaches the Sun, and as been detected in the coma of comets observed at relatively large distances from the Sun \citep[e.g. ][]{Fitzsimmons1996}.
%
The relative abundance in the coma of most comets of \ce{CN/OH}$\simeq500$ \citep{AHearn1995}, together with its high fluorescence scattering efficiency factor of $\sim$ 2.6 photons/sec/molecule at 1~au from the Sun and $\dot{r}\simeq 0$ \citep{Schleicher2010}, gives rise to a clear signature of a cometary gas comae.
%
Historically, the \ce{CN} was believed to be created from photodissociation of \ce{HCN}, which is present in abundances of the order of 1\% relative to water in cometary ices.
%
High-spatial resolution sub-mm mapping of \ce{HCN} distributions supports a nuclear source in some comets \citep{Cordiner2014}.
%
However, it is now known that in many comets a substantial fraction of \ce{CN} is released from sub-micron dust grains in the inner coma of comets \citep[e.g. ][]{Fray2005}. 

The \ce{C2} (0-0) band at $\lambda\simeq 5160$\AA \ may be of comparable flux, but has a lower contrast with the underlying dust grain continuum due to its wider bandwidth.
%
The \ce{OH} (0-0) band at $\lambda\simeq 3080$ \AA \ generally has higher flux, but unless observed from space the high terrestrial atmospheric absorption severely diminishes the observed flux.
%
Strong \ce{CO+} emission bands have also been observed in a number of comets (the strongest ones are the (3-0) and (2-0) $\mathrm{A^2\Pi-X^2\Sigma}$) \hl{REFS}).
%
However this is normally significantly weaker than the observed \ce{CN} emission, with only \hl{number of comets} comets.

Given the ubiquitous signature of \ce{CN} emission in Solar system comets, we have embarked on an archival search for this species in the \ac{feb} around Beta Pic.
%
Detection of this species would provide a solid link between cometary bodies in our Solar system and the \bp{} system.

\section{Data description and analysis}

For the study in this paper, we use all the publicly available data from the \ac{harps} and the ESO 3.6m telescope in Chile.
%
In addition, we complement this with observations with the \ac{uves} at the Very Large Telescope in Chile. 

\subsection{HARPS observations}

We collated all publicly available HARPS observations of \bp{} obtained between October 2003 and \textcolor{red}{March 2020}.
%
Although these observations cover a large baseline, the cadence between visits varies between 1 day and $\sim$2.2 years.
%
Similarly, the number of observations per epoch spans a wide range, between \textcolor{red}{1 and 356} exposures, while the exposure times vary between \textcolor{red}{12 and 900} seconds. This results in a SNR range per exposure between \textcolor{red}{NNNN and MMMM}, and a nightly SNR between \textcolor{red}{NNNN and MMMM}.
%
To avoid potential edge effects from order merging,  we used the E2DS files downloaded directly from the ESO archive for our analysis.

\textcolor{red}{Q: Do we want to put an observing log in the appendix?}
\textcolor{red}{Q: Do we want to attempt to use the S1D, as they may have less issues with wiggles if the blaze profile was removed.}
\subsection{UVES observations}

To supplement the HARPS observations, we use 160 epochs of \ac{uves} observations obtained between 01 April, 2017 and April 18, 2018 under programme 599.C-0751.
%
The details for these observations are given in \cite{vansluijs2019}, and summarised here.
%
These observations were all obtained using the same instrument setup.
%
We used the standard 437+760 mode of \ac{uves}, which uses the \#2 dichroic to allow simultaneous observations in both the red and blue arm.
%
In this mode, the CD\#2 and CD\#4 cross-dispersers are used in the blue and red-arm, respectively, providing a simultaneous wavelength coverage of $\sim 3730$ \AA{} to $\sim 5000$ \AA{} in the blue arm, and $\sim 5650$ \AA{} to $\sim 9560$ \AA{} in the red arm.
%
For the remainder of the paper, we will only focus on the data from the blue arm of \ac{uves}, as this covers the strong \ce{CN} (0-0) band-head seen in Solar system comets.
%
For the blue arm, a 0.3''-wide slit was used to obtain the highest possible resolving power ($\sim$90,000 before an intervention on the instrument in October 2017, and $\sim$100,000 after the intervention). 

The data were reduced using the ESO \ac{uves} pipeline version \textcolor{red}{6.1.6} through ESO Reflex tool in order to provide as homogeneous a dataset as possible. The flat-field was applied on a pixel-by-pixel basis.\textcolor{red}{The pipeline was set to output the intermediate products and the blaze-corrected spectra before merging were used for our analysis, as the order-merging could introduce systematics in the overlap region. Optimal extraction was performed, using an empirical profile}

\subsection{Initial data processing}
Starting from the 2d spectra, the data from both instruments was processed in similar ways.
%
Each order was treated separately, with the main focus on orders \textcolor{red}{NN, MM and OO} for HARPS and \textcolor{red}{NN, MM, OO} for UVES as those orders contain the main CN bandhead as well as the \ion{Ca}{ii} H \& K lines, which are used to identify the exocomets.
%
All the spectra were transformed into the stellar rest-frame and onto a common wavelength grid using linear interpolation. 

The number of observations varies from night to night, from a \textcolor{red}{single} spectrum to over \textcolor{red}{three hundred} spectra.
%
An average of all the spectra in one night was made for each of the nights of observations, after correcting for variations in the effective blaze profile (e.g., due to instrumental changes, differential slit losses, atmospheric absorption etc.), as outlined below. Note that after this correction, the HARPS data still have the overall instrumental blaze-profile, while the 2d-spectra from UVES were already corrected for this.

\subsubsection{Effective Blaze function correction}

One of the nights with a high signal to noise was chosen to act as an initial estimate for the blaze function (the night of \textcolor{red}{DATE} for HARPS and \textcolor{red}{DATE} for UVES). The individual spectra in this night were averaged together to further boost the signal-to-noise ratio. Subsequently, each individual spectrum was divided by this reference spectrum to highlight the changes in the effective profile. This ratio spectrum was subsequently binned using a bin-width of \textcolor{red}{101} pixels using a robust algorithm to remove outliers within each bin. This binned ratio was than fit using a \textcolor{red}{second} order polynomial. The original spectrum was finally divided by this polynomial to correct for the changes.


%his spectrum was binned in the spectral direction to remove both narrow absorption lines and the rotationally broadened stellar chromospheric lines, but preserve the shape of the blaze function.
%


\subsubsection{Combining the spectra}
After the blaze-variation corrections, all the spectra within a night are combined in order to further boost the signal-to-noise and improve the detection limits.

For the spectra within a given night:

\begin{itemize}
\item We median combine all the nights together.

\item Bin the spectrum and clip it, interpolate using a cubic spline.

\item Divide this interpolated cubic spline into each of the nightly spectra.

\end{itemize}

This removes the blaze function, but there remain low spatial frequency variations between different nights, due to the changing slit illumination as the star moves across the slit aperture.

We then fit a second order polynomial to each individual night and divide this out, resulting in a flattened spectrum for each night.


Subsequently, we applied a correction for variations in the instrumental blaze by selecting a reference spectrum and dividing all the spectra by this reference. The ratio was then binned, removing outliers, and a polynomial was fit to the data.
%
For \ac{uves} we used a Nth degree polynomial, while for \ac{harps} we used a 2nd degree polynomial.
%
This polynomial was then divided from the spectrum under consideration. We note that the blaze correction will also remove non-instrumental effects, including variations in atmospheric transmission and seeing.

The blaze-corrected spectra taken within a night were averaged together [Should we use a weighted average?].
%
We note that for \ac{harps}, the 2d-unmerged spectra still contain the overal instrumental blaze-profile, which is (partially) corrected for by using a dedicated blaze-frame obtained from the archive, which removes most, but not all, of this profile. 

\section{Searches for cometary species}

After the initial data analysis, we proceed to with our search for lines from \ce{CN}.
%
Although we are interested in lines from the exocomets, we first co-add the data from all the individual epochs, in the stellar rest-frame, to obtain a deep search for circumstellar \ce{CN} absorption.
%
These coadded spectra for \ac{harps} and \ac{uves} are shown in \hl{Figure}.
%
No obvious feature jumps out, and we can rule out circumstellar \ce{CN} down to \hl{NNNN}.

Since the exocomets have a clear radial velocity with respect to \bp{}, and since each exocomet will have its own distinct velocity, we selected the NN epochs with the strongest exocometary features, a seen in the \ion{Ca}{ii} H \& K lines, as well as the MM epochs with the weakest cometary features.
%
The spectra with only weak exocometary features were co-averaged in the stellar rest frame in order to create a stellar master spectrum. This master-spectrum was divided out of all the remaining spectra in order to create relative spectra. 
%
The relative velocities of the exocomets were determined by ......., and the spectra were shifted to the rest-frames of these comments, before being combined.
%
The combined exocomet aligned spectra for both the \ce{Ca} H\&K lines and \ce{CN} band are shown in Fig.~\ref{fig:coadd_exocomets}. As can be seen no signal is visibly present.
%
[Do we need to add a correction to account for any radiation pressure, if so, is there a way to estimate that?]

[Do we want to do a simple CCF to see if we can combine the individual lines?]

We have theoretical spectra of CN shown in Figure~\ref{fig:CN_theory}.

\begin{figure}
    \begin{centering}
        \includegraphics[width=\columnwidth]{figures/two_CN_temps.pdf}
        \caption{Model spectrum for CN.}
        \label{fig:CN_theory}
        \script{plot_two_CN_temps.py}
    \end{centering}
\end{figure}



\begin{figure}
    \begin{centering}
        \includegraphics[width=\columnwidth]{figures/ccf.pdf}
        \caption{Cross correlation for different number densities of CN.}
        \label{fig:ccf}
        \script{plot_ccf.py}
    \end{centering}
\end{figure}


\begin{figure}
    \begin{centering}
        \includegraphics[width=\columnwidth]{figures/ccf_mean.pdf}
        \caption{Cross correlation for different number densities of CN averaged over many spectra.}
        \label{fig:ccf_mean}
        \script{plot_ccf_mean.py}
    \end{centering}
\end{figure}




\subsection{\ce{CN}}

% \begin{figure}
%     \begin{centering}
%         \includegraphics[width=\columnwidth]{figures/HARPS_CN_grid.pdf}
%         \caption{Confidence intervals for the signal to noise of \ce{CN} spectra using \ac{harps} spectra of \bp{}.}
%         \label{fig:HARPS_CN_grid}
%         \script{plot_HARPS_CN_grid.py}
%     \end{centering}
% \end{figure}

%\subsection{Other Species? CO$^+$, C$_2$?}

\section{Discussion}

\section{Conclusions}


\begin{acknowledgements}

The authors thank the Lorentz Centre at Leiden University for organising the workshop ``Exocomets: Understanding the Composition of Planetary Building Blocks'' which led to the study in this paper. 
%
Part of this work was supported by the German \emph{Deut\-sche For\-schungs\-ge\-mein\-schaft, DFG\/} project number Ts~17/2--1.
%
We made use of the {\tt Python} programming language \citep{rossum1995} and the open-source {\tt Python} packages {\tt numpy} \citep{walt2011}, {\tt matplotlib} \citep{hunter2007}, {\tt astropy} \citep{astropy2013}.
%      
An online repository with materials used in this work is available at \url{https://github.com/mkenworthy/}
\end{acknowledgements}


%-------------------------------------------------------------------

\bibliographystyle{aa}
\bibliography{bib.bib}

%\begin{thebibliography}{}

%  \bibitem[1966]{baker} Baker, N. 1966,
%      in Stellar Evolution,
%      ed.\ R. F. Stein,\& A. G. W. Cameron
%      (Plenum, New York) 333

%   \bibitem[1988]{balluch} Balluch, M. 1988,
%      A\&A, 200, 58

%   \bibitem[1980]{cox} Cox, J. P. 1980,
%      Theory of Stellar Pulsation
%      (Princeton University Press, Princeton) 165

 %  \bibitem[1969]{cox69} Cox, A. N.,\& Stewart, J. N. 1969,
 %     Academia Nauk, Scientific Information 15, 1

 %  \bibitem[1980]{mizuno} Mizuno H. 1980,
 %     Prog. Theor. Phys., 64, 544
   
%   \bibitem[1987]{tscharnuter} Tscharnuter W. M. 1987,
%      A\&A, 188, 55
  
%   \bibitem[1992]{terlevich} Terlevich, R. 1992, in ASP Conf. Ser. 31, 
%      Relationships between Active Galactic Nuclei and Starburst Galaxies, 
%      ed. A. V. Filippenko, 13

%   \bibitem[1980a]{yorke80a} Yorke, H. W. 1980a,
%      A\&A, 86, 286

%   \bibitem[1997]{zheng} Zheng, W., Davidsen, A. F., Tytler, D. \& Kriss, G. A.
%      1997, preprint
%\end{thebibliography}

\end{document}

%%%%%%%%%%%%%%%%%%%%%%%%%%%%%%%%%%%%%%%%%%%%%%%%%%%%%%%%%%%%%%
Examples for figures using graphicx
A guide "Using Imported Graphics in LaTeX2e"  (Keith Reckdahl)
is available on a lot of LaTeX public servers or ctan mirrors.
The file is : epslatex.pdf 
%%%%%%%%%%%%%%%%%%%%%%%%%%%%%%%%%%%%%%%%%%%%%%%%%%%%%%%%%%%%%%

%_____________________________________________________________
%                 A figure as large as the width of the column
%-------------------------------------------------------------
   \begin{figure}
   \centering
   \includegraphics[width=\hsize]{empty.eps}
      \caption{Vibrational stability equation of state
               $S_{\mathrm{vib}}(\lg e, \lg \rho)$.
               $>0$ means vibrational stability.
              }
         \label{FigVibStab}
   \end{figure}
%
%_____________________________________________________________
%                                    One column rotated figure
%-------------------------------------------------------------
   \begin{figure}
   \centering
   \includegraphics[angle=-90,width=3cm]{empty.eps}
      \caption{Vibrational stability equation of state
               $S_{\mathrm{vib}}(\lg e, \lg \rho)$.
               $>0$ means vibrational stability.
              }
         \label{FigVibStab}
   \end{figure}
%
%_____________________________________________________________
%                        Figure with caption on the right side 
%-------------------------------------------------------------
   \begin{figure}
   \sidecaption
   \includegraphics[width=3cm]{empty.eps}
      \caption{Vibrational stability equation of state
               $S_{\mathrm{vib}}(\lg e, \lg \rho)$.
               $>0$ means vibrational stability.
              }
         \label{FigVibStab}
   \end{figure}
%
%_____________________________________________________________
%
%_____________________________________________________________
%                                Figure with a new BoundingBox 
%-------------------------------------------------------------
   \begin{figure}
   \centering
   \includegraphics[bb=10 20 100 300,width=3cm,clip]{empty.eps}
      \caption{Vibrational stability equation of state
               $S_{\mathrm{vib}}(\lg e, \lg \rho)$.
               $>0$ means vibrational stability.
              }
         \label{FigVibStab}
   \end{figure}
%
%_____________________________________________________________
%
%_____________________________________________________________
%                                      The "resizebox" command 
%-------------------------------------------------------------
   \begin{figure}
   \resizebox{\hsize}{!}
            {\includegraphics[bb=10 20 100 300,clip]{empty.eps}
      \caption{Vibrational stability equation of state
               $S_{\mathrm{vib}}(\lg e, \lg \rho)$.
               $>0$ means vibrational stability.
              }
         \label{FigVibStab}
   \end{figure}
%
%______________________________________________________________
%
%_____________________________________________________________
%                                             Two column Figure 
%-------------------------------------------------------------
   \begin{figure*}
   \resizebox{\hsize}{!}
            {\includegraphics[bb=10 20 100 300,clip]{empty.eps}
      \caption{Vibrational stability equation of state
               $S_{\mathrm{vib}}(\lg e, \lg \rho)$.
               $>0$ means vibrational stability.
              }
         \label{FigVibStab}
   \end{figure*}
%
%______________________________________________________________
%
%_____________________________________________________________
%                                             Simple A&A Table
%_____________________________________________________________
%
\begin{table}
\caption{Nonlinear Model Results}             % title of Table
\label{table:1}      % is used to refer this table in the text
\centering                          % used for centering table
\begin{tabular}{c c c c}        % centered columns (4 columns)
\hline\hline                 % inserts double horizontal lines
HJD & $E$ & Method\#2 & Method\#3 \\    % table heading 
\hline                        % inserts single horizontal line
   1 & 50 & $-837$ & 970 \\      % inserting body of the table
   2 & 47 & 877    & 230 \\
   3 & 31 & 25     & 415 \\
   4 & 35 & 144    & 2356 \\
   5 & 45 & 300    & 556 \\ 
\hline                                   %inserts single line
\end{tabular}
\end{table}
%
%_____________________________________________________________
%                                             Two column Table 
%_____________________________________________________________
%
\begin{table*}
\caption{Nonlinear Model Results}             
\label{table:1}      
\centering          
\begin{tabular}{c c c c l l l }     % 7 columns 
\hline\hline       
                      % To combine 4 columns into a single one 
HJD & $E$ & Method\#2 & \multicolumn{4}{c}{Method\#3}\\ 
\hline                    
   1 & 50 & $-837$ & 970 & 65 & 67 & 78\\  
   2 & 47 & 877    & 230 & 567& 55 & 78\\
   3 & 31 & 25     & 415 & 567& 55 & 78\\
   4 & 35 & 144    & 2356& 567& 55 & 78 \\
   5 & 45 & 300    & 556 & 567& 55 & 78\\
\hline                  
\end{tabular}
\end{table*}
%
%-------------------------------------------------------------
%                                          Table with notes 
%-------------------------------------------------------------
%
% A single note
\begin{table}
\caption{\label{t7}Spectral types and photometry for stars in the
  region.}
\centering
\begin{tabular}{lccc}
\hline\hline
Star&Spectral type&RA(J2000)&Dec(J2000)\\
\hline
69           &B1\,V     &09 15 54.046 & $-$50 00 26.67\\
49           &B0.7\,V   &*09 15 54.570& $-$50 00 03.90\\
LS~1267~(86) &O8\,V     &09 15 52.787&11.07\\
24.6         &7.58      &1.37 &0.20\\
\hline
LS~1262      &B0\,V     &09 15 05.17&11.17\\
MO 2-119     &B0.5\,V   &09 15 33.7 &11.74\\
LS~1269      &O8.5\,V   &09 15 56.60&10.85\\
\hline
\end{tabular}
\tablefoot{The top panel shows likely members of Pismis~11. The second
panel contains likely members of Alicante~5. The bottom panel
displays stars outside the clusters.}
\end{table}
%
% More notes
%
\begin{table}
\caption{\label{t7}Spectral types and photometry for stars in the
  region.}
\centering
\begin{tabular}{lccc}
\hline\hline
Star&Spectral type&RA(J2000)&Dec(J2000)\\
\hline
69           &B1\,V     &09 15 54.046 & $-$50 00 26.67\\
49           &B0.7\,V   &*09 15 54.570& $-$50 00 03.90\\
LS~1267~(86) &O8\,V     &09 15 52.787&11.07\tablefootmark{a}\\
24.6         &7.58\tablefootmark{1}&1.37\tablefootmark{a}   &0.20\tablefootmark{a}\\
\hline
LS~1262      &B0\,V     &09 15 05.17&11.17\tablefootmark{b}\\
MO 2-119     &B0.5\,V   &09 15 33.7 &11.74\tablefootmark{c}\\
LS~1269      &O8.5\,V   &09 15 56.60&10.85\tablefootmark{d}\\
\hline
\end{tabular}
\tablefoot{The top panel shows likely members of Pismis~11. The second
panel contains likely members of Alicante~5. The bottom panel
displays stars outside the clusters.\\
\tablefoottext{a}{Photometry for MF13, LS~1267 and HD~80077 from
Dupont et al.}
\tablefoottext{b}{Photometry for LS~1262, LS~1269 from
Durand et al.}
\tablefoottext{c}{Photometry for MO2-119 from
Mathieu et al.}
}
\end{table}
%
%-------------------------------------------------------------
%                                       Table with references 
%-------------------------------------------------------------
%
\begin{table*}[h]
 \caption[]{\label{nearbylistaa2}List of nearby SNe used in this work.}
\begin{tabular}{lccc}
 \hline \hline
  SN name &
  Epoch &
 Bands &
  References \\
 &
  (with respect to $B$ maximum) &
 &
 \\ \hline
1981B   & 0 & {\it UBV} & 1\\
1986G   &  $-$3, $-$1, 0, 1, 2 & {\it BV}  & 2\\
1989B   & $-$5, $-$1, 0, 3, 5 & {\it UBVRI}  & 3, 4\\
1990N   & 2, 7 & {\it UBVRI}  & 5\\
1991M   & 3 & {\it VRI}  & 6\\
\hline
\noalign{\smallskip}
\multicolumn{4}{c}{ SNe 91bg-like} \\
\noalign{\smallskip}
\hline
1991bg   & 1, 2 & {\it BVRI}  & 7\\
1999by   & $-$5, $-$4, $-$3, 3, 4, 5 & {\it UBVRI}  & 8\\
\hline
\noalign{\smallskip}
\multicolumn{4}{c}{ SNe 91T-like} \\
\noalign{\smallskip}
\hline
1991T   & $-$3, 0 & {\it UBVRI}  &  9, 10\\
2000cx  & $-$3, $-$2, 0, 1, 5 & {\it UBVRI}  & 11\\ %
\hline
\end{tabular}
\tablebib{(1)~\citet{branch83};
(2) \citet{phillips87}; (3) \citet{barbon90}; (4) \citet{wells94};
(5) \citet{mazzali93}; (6) \citet{gomez98}; (7) \citet{kirshner93};
(8) \citet{patat96}; (9) \citet{salvo01}; (10) \citet{branch03};
(11) \citet{jha99}.
}
\end{table}
%_____________________________________________________________
%                      A rotated Two column Table in landscape  
%-------------------------------------------------------------
\begin{sidewaystable*}
\caption{Summary for ISOCAM sources with mid-IR excess 
(YSO candidates).}\label{YSOtable}
\centering
\begin{tabular}{crrlcl} 
\hline\hline             
ISO-L1551 & $F_{6.7}$~[mJy] & $\alpha_{6.7-14.3}$ 
& YSO type$^{d}$ & Status & Comments\\
\hline
  \multicolumn{6}{c}{\it New YSO candidates}\\ % To combine 6 columns into a single one
\hline
  1 & 1.56 $\pm$ 0.47 & --    & Class II$^{c}$ & New & Mid\\
  2 & 0.79:           & 0.97: & Class II ?     & New & \\
  3 & 4.95 $\pm$ 0.68 & 3.18  & Class II / III & New & \\
  5 & 1.44 $\pm$ 0.33 & 1.88  & Class II       & New & \\
\hline
  \multicolumn{6}{c}{\it Previously known YSOs} \\
\hline
  61 & 0.89 $\pm$ 0.58 & 1.77 & Class I & \object{HH 30} & Circumstellar disk\\
  96 & 38.34 $\pm$ 0.71 & 37.5& Class II& MHO 5          & Spectral type\\
\hline
\end{tabular}
\end{sidewaystable*}
%_____________________________________________________________
%                      A rotated One column Table in landscape  
%-------------------------------------------------------------
\begin{sidewaystable}
\caption{Summary for ISOCAM sources with mid-IR excess 
(YSO candidates).}\label{YSOtable}
\centering
\begin{tabular}{crrlcl} 
\hline\hline             
ISO-L1551 & $F_{6.7}$~[mJy] & $\alpha_{6.7-14.3}$ 
& YSO type$^{d}$ & Status & Comments\\
\hline
  \multicolumn{6}{c}{\it New YSO candidates}\\ % To combine 6 columns into a single one
\hline
  1 & 1.56 $\pm$ 0.47 & --    & Class II$^{c}$ & New & Mid\\
  2 & 0.79:           & 0.97: & Class II ?     & New & \\
  3 & 4.95 $\pm$ 0.68 & 3.18  & Class II / III & New & \\
  5 & 1.44 $\pm$ 0.33 & 1.88  & Class II       & New & \\
\hline
  \multicolumn{6}{c}{\it Previously known YSOs} \\
\hline
  61 & 0.89 $\pm$ 0.58 & 1.77 & Class I & \object{HH 30} & Circumstellar disk\\
  96 & 38.34 $\pm$ 0.71 & 37.5& Class II& MHO 5          & Spectral type\\
\hline
\end{tabular}
\end{sidewaystable}
%
%_____________________________________________________________
%                              Table longer than a single page  
%-------------------------------------------------------------
% All long tables will be placed automatically at the end, after 
%                                        \end{thebibliography}
%
\begin{longtab}
\begin{longtable}{lllrrr}
\caption{\label{kstars} Sample stars with absolute magnitude}\\
\hline\hline
Catalogue& $M_{V}$ & Spectral & Distance & Mode & Count Rate \\
\hline
\endfirsthead
\caption{continued.}\\
\hline\hline
Catalogue& $M_{V}$ & Spectral & Distance & Mode & Count Rate \\
\hline
\endhead
\hline
\endfoot
%%
Gl 33    & 6.37 & K2 V & 7.46 & S & 0.043170\\
Gl 66AB  & 6.26 & K2 V & 8.15 & S & 0.260478\\
Gl 68    & 5.87 & K1 V & 7.47 & P & 0.026610\\
         &      &      &      & H & 0.008686\\
Gl 86 
\footnote{Source not included in the HRI catalog. See Sect.~5.4.2 for details.}
         & 5.92 & K0 V & 10.91& S & 0.058230\\
\end{longtable}
\end{longtab}
%
%_____________________________________________________________
%                              Table longer than a single page
%                                             and in landscape 
%  In the preamble, use:       \usepackage{lscape}
%-------------------------------------------------------------
% All long tables will be placed automatically at the end, after
%                                        \end{thebibliography}
%
\begin{longtab}
\begin{landscape}
\begin{longtable}{lllrrr}
\caption{\label{kstars} Sample stars with absolute magnitude}\\
\hline\hline
Catalogue& $M_{V}$ & Spectral & Distance & Mode & Count Rate \\
\hline
\endfirsthead
\caption{continued.}\\
\hline\hline
Catalogue& $M_{V}$ & Spectral & Distance & Mode & Count Rate \\
\hline
\endhead
\hline
\endfoot
%%
Gl 33    & 6.37 & K2 V & 7.46 & S & 0.043170\\
Gl 66AB  & 6.26 & K2 V & 8.15 & S & 0.260478\\
Gl 68    & 5.87 & K1 V & 7.47 & P & 0.026610\\
         &      &      &      & H & 0.008686\\
Gl 86
\footnote{Source not included in the HRI catalog. See Sect.~5.4.2 for details.}
         & 5.92 & K0 V & 10.91& S & 0.058230\\
\end{longtable}
\end{landscape}
\end{longtab}
%
% Online Material
%_____________________________________________________________
%        Online appendices have to be placed at the end, after
%                                        \end{thebibliography}
%-------------------------------------------------------------
\end{thebibliography}


\Online

\begin{appendix} %First online appendix
\section{Background galaxy number counts and shear noise-levels}
Because the optical images used in this analysis...

\begin{figure*}
\centering
\includegraphics[width=16.4cm,clip]{1787f24.ps}
\caption{Plotted above...}
\label{appfig}
\end{figure*}

Because the optical images...
\end{appendix}

\begin{appendix} %Second online appendix
These studies, however, have faced...
\end{appendix}

\end{document}
%
%_____________________________________________________________
%        Some tables or figures are in the printed version and
%                      some are only in the electronic version
%-------------------------------------------------------------
%
% Leave all the tables or figures in the text, at their right place 
% and use the commands \onlfig{} and \onltab{}. These elements
% will be automatically placed at the end, in the section
% Online material.

\documentclass{aa}
...
\begin{document}
text of the paper...
\begin{figure*}%f1
\includegraphics[width=10.9cm]{1787f01.eps}
\caption{Shown in greyscale is a...}
\label{cl12301}}
\end{figure*}
...
from the intrinsic ellipticity distribution.
% Figure 2 available electronically only
\onlfig{
\begin{figure*}%f2
\includegraphics[width=11.6cm]{1787f02.eps}
\caption {Shown in greyscale...}
\label{cl1018}
\end{figure*}
}

% Figure 3 available electronically only
\onlfig{
\begin{figure*}%f3
\includegraphics[width=11.2cm]{1787f03.eps}
\caption{Shown in panels...}
\label{cl1059}
\end{figure*}
}

\begin{figure*}%f4
\includegraphics[width=10.9cm]{1787f04.eps}
\caption{Shown in greyscale is...}
\label{cl1232}}
\end{figure*}

\begin{table}%t1
\caption{Complexes characterisation.}\label{starbursts}
\centering
\begin{tabular}{lccc}
\hline \hline
Complex & $F_{60}$ & 8.6 &  No. of  \\
...
\hline
\end{tabular}
\end{table}
The second method produces...

% Figure 5 available electronically only
\onlfig{
\begin{figure*}%f5
\includegraphics[width=11.2cm]{1787f05.eps}
\caption{Shown in panels...}
\label{cl1238}}
\end{figure*}
}

As can be seen, in general the deeper...
% Table 2 available electronically only
\onltab{
\begin{table*}%t2
\caption{List of the LMC stellar complexes...}\label{Properties}
\centering
\begin{tabular}{lccccccccc}
\hline  \hline
Stellar & RA & Dec & ...
...
\hline
\end{tabular}
\end{table*}
}

% Table 3 available electronically only
\onltab{
\begin{table*}%t3
\caption{List of the derived...}\label{IrasFluxes}
\centering
\begin{tabular}{lcccccccccc}
\hline \hline
Stellar & $f12$ & $L12$ &...
...
\hline
\end{tabular}
\end{table*}
}
%
%-------------------------------------------------------------
%     For the online material, table longer than a single page
%                 In the preamble for landscape case, use : 
%                                          \usepackage{lscape}
%-------------------------------------------------------------
\documentclass{aa}
\usepackage[varg]{txfonts}
\usepackage{graphicx}
\usepackage{lscape}

\begin{document}
text of the paper
% Table will be print automatically at the end, in the section Online material.
\onllongtab{
\begin{longtable}{lrcrrrrrrrrl}
\caption{Line data and abundances ...}\\
\hline
\hline
Def & mol & Ion & $\lambda$ & $\chi$ & $\log gf$ & N & e &  rad & $\delta$ & $\delta$ 
red & References \\
\hline
\endfirsthead
\caption{Continued.} \\
\hline
Def & mol & Ion & $\lambda$ & $\chi$ & $\log gf$ & B & C &  rad & $\delta$ & $\delta$ 
red & References \\
\hline
\endhead
\hline
\endfoot
\hline
\endlastfoot
A & CH & 1 &3638 & 0.002 & $-$2.551 &  &  &  & $-$150 & 150 &  Jorgensen et al. (1996) \\                    
\end{longtable}
}% End onllongtab

% Or for landscape, large table:

\onllongtab{
\begin{landscape}
\begin{longtable}{lrcrrrrrrrrl}
...
\end{longtable}
\end{landscape}
}% End onllongtab